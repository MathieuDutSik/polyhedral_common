\documentclass[11pt,a4paper]{article}
\usepackage[utf8]{inputenc}
\usepackage{amsmath}
\usepackage{amsfonts}
\usepackage{amssymb}
\usepackage{graphicx}
\usepackage{hyperref}
\usepackage{listings}
\usepackage{xcolor}
\usepackage{geometry}
\geometry{margin=1in}

\lstset{
    basicstyle=\ttfamily\small,
    breaklines=true,
    frame=single,
    language=C++,
    commentstyle=\color{gray},
    keywordstyle=\color{blue},
    stringstyle=\color{red}
}

\title{Polyhedral Common Library\\
\large Computational Tools for Polytopes, Lattices, and Quadratic Forms\\
Documentation}
\author{Mathieu Dutour Sikiric}
\date{\today}

\begin{document}

\maketitle

\tableofcontents
\newpage

\section{Introduction}

The Polyhedral Common Library is a comprehensive set of computational tools designed for advanced mathematical computations involving polytopes, quadratic forms, and lattices. The library is optimized for extreme performance to tackle record-breaking problems in computational mathematics.

\subsection{Key Features}

\begin{itemize}
    \item High-performance algorithms for polyhedral computations
    \item Extensive group theory functionality for polytope symmetries
    \item Lattice theory implementations (Delaunay domains, canonicalization)
    \item Copositivity and strict copositivity testing
    \item Short vector computations and enumeration
    \item Perfect form computations
    \item Indefinite and Lorentzian form reductions
    \item Support for multiple arithmetic types (exact rational, algebraic numbers, safe integers)
\end{itemize}

\subsection{Design Philosophy}

The library follows these core principles:
\begin{itemize}
    \item \textbf{Performance-First}: All algorithms are optimized for speed using group theory and polyhedral structures
    \item \textbf{Modularity}: Independent subdirectories for different functionalities
    \item \textbf{Flexibility}: Template-based design supporting multiple arithmetic types
    \item \textbf{Exactness}: Preference for exact arithmetic to avoid numerical errors
\end{itemize}

\section{Project Organization}

The project is organized into several independent subdirectories, each focusing on a specific mathematical domain:

\begin{description}
    \item[\texttt{src\_group}] Group theory computations for polytopes
    \item[\texttt{src\_poly}] Polyhedral computations (dual descriptions, facets, vertices)
    \item[\texttt{src\_latt}] Lattice computations (Delaunay, canonicalization, isomorphism)
    \item[\texttt{src\_dualdesc}] Dual description algorithms for serial computers
    \item[\texttt{src\_copos}] Copositivity and complete positivity testing
    \item[\texttt{src\_short}] Short vector enumeration and computations
    \item[\texttt{src\_perfect}] Perfect form computations
    \item[\texttt{src\_ctype}] C-type enumeration and classification
    \item[\texttt{src\_indefinite}] Indefinite quadratic form reduction
    \item[\texttt{src\_lorentzian}] Lorentzian/hyperbolic form algorithms (Vinberg, Edgewalk)
    \item[\texttt{src\_isotropy}] Isotropic vector finding and reduction
    \item[\texttt{src\_poincare\_polyhedron}] Poincaré polyhedron computations
    \item[\texttt{src\_sparse\_solver}] Sparse linear system solvers
\end{description}

\section{Module Documentation}

\subsection{Group Theory Module (\texttt{src\_group})}

This module provides comprehensive functionality for computing and manipulating groups acting on polytopes and lattices.

\subsubsection{Core Functionalities}

\paragraph{Matrix Group Operations}
The module implements efficient representations and operations on matrix groups:

\begin{itemize}
    \item \textbf{Group generation}: Constructing groups from generators
    \item \textbf{Orbit enumeration}: Computing orbits under group actions
    \item \textbf{Stabilizer computation}: Finding stabilizer subgroups
    \item \textbf{Double coset enumeration}: Computing double coset decompositions
    \item \textbf{Group simplification}: Reducing group representations
\end{itemize}

\paragraph{Key Functions}

\begin{lstlisting}
// Write matrix group to file
template <typename T>
void write_matrix_group(
    std::vector<MyMatrix<T>> const &list_mat,
    std::string const &context);

// Compute rational invariant of matrix group
template <typename T>
size_t GetRationalInvariant(
    std::vector<MyMatrix<T>> const &ListGen);

// LLL reduction for matrix groups
template <typename T, typename Tint, typename Thelper>
std::pair<std::vector<MyMatrix<T>>, MyMatrix<Tint>>
LLLMatrixGroupReduction(
    Thelper const &helper,
    std::vector<MyMatrix<T>> const &ListMatr,
    std::ostream &os);

// Compute L1 norm of matrix group
template <typename T, typename Thelper>
T L1normMatrixGroup(
    Thelper const &helper,
    std::vector<MyMatrix<T>> const &ListMatr);
\end{lstlisting}

\subsubsection{Key Header Files}

\begin{description}
    \item[\texttt{MatrixGroupBasic.h}] Basic matrix group operations and utilities
    \item[\texttt{MatrixGroup.h}] Core matrix group data structures
    \item[\texttt{OrbitEnumeration.h}] Orbit computation algorithms
    \item[\texttt{GRP\_DoubleCoset.h}] Double coset enumeration
    \item[\texttt{MatrixGroupSimplification.h}] Group simplification algorithms
\end{description}

\subsection{Polyhedral Module (\texttt{src\_poly})}

This module handles all aspects of polyhedral computations, including dual descriptions, face enumeration, and linear programming.

\subsubsection{Core Functionalities}

\paragraph{Basic Operations}
\begin{itemize}
    \item \textbf{Row selection}: Extracting submatrices based on face specifications
    \item \textbf{Facet testing}: Verifying facet inequalities
    \item \textbf{Face enumeration}: Computing all faces of a polytope
    \item \textbf{Dual description}: Converting between vertex and facet representations
\end{itemize}

\paragraph{Key Functions}

\begin{lstlisting}
// Select rows from matrix based on face
template <typename T>
MyMatrix<T> SelectRow(
    MyMatrix<T> const &TheMat,
    Face const &eList);

// Select rows and drop a column
template <typename T>
MyMatrix<T> SelectRowDropColumnFace(
    MyMatrix<T> const &TheMat,
    Face const &eList,
    int const &idx_drop);

// Test if a face defines a valid facet
template <typename T>
bool TestFacetInequality(
    MyMatrix<T> const &EXT,
    Face const &eList);
\end{lstlisting}

\subsubsection{Algorithms}

\begin{description}
    \item[cddlib integration] Interface to cddlib for dual description
    \item[Direct dual description] Custom implementation for specific cases
    \item[Redundancy removal] Identifying and removing redundant constraints
    \item[Linear programming] Solving linear programs over polytopes
\end{description}

\subsection{Lattice Module (\texttt{src\_latt})}

The lattice module provides advanced algorithms for lattice computations, particularly focusing on positive definite quadratic forms.

\subsubsection{Core Functionalities}

\paragraph{Shortest Vectors}
The module implements efficient algorithms for computing shortest vectors in lattices:

\begin{lstlisting}
// Data structure for shortest vectors
template <typename T, typename Tint>
struct Tshortest {
    T min;                  // Minimum norm
    MyMatrix<Tint> SHV;     // Shortest vectors
};

// Select shortest vectors from a set
template <typename T, typename Tint>
Tshortest<T, Tint> SelectShortestVector(
    MyMatrix<T> const &eMat,
    MyMatrix<Tint> const &SHV);

// Apply transformation to shortest vectors
template <typename T, typename Tint>
Tshortest<T, Tint> apply_transformation(
    Tshortest<T, Tint> const &rec_shv,
    MyMatrix<Tint> const &P);

// Get half of the shortest vectors (exploit symmetry)
template <typename T, typename Tint>
Tshortest<T, Tint> shortest_get_half(
    Tshortest<T, Tint> const &rec_shv);
\end{lstlisting}

\paragraph{Delaunay Computations}
\begin{itemize}
    \item \textbf{Delaunay domain enumeration}: Computing Delaunay polytopes
    \item \textbf{Isodelaunay domains}: Finding isomorphism classes
    \item \textbf{Fundamental domains}: Computing fundamental Delaunay domains
\end{itemize}

\paragraph{Canonicalization}
\begin{itemize}
    \item \textbf{Matrix canonicalization}: Computing canonical forms of positive definite matrices
    \item \textbf{Automorphism groups}: Finding symmetries of lattices
    \item \textbf{Isomorphism testing}: Determining lattice equivalence
\end{itemize}

\subsubsection{Key Header Files}

\begin{description}
    \item[\texttt{LatticeDefinitions.h}] Core data structures and basic operations
    \item[\texttt{LatticeDelaunay.h}] Delaunay domain computations
    \item[\texttt{LatticeStabEquiCan.h}] Stabilizer, equivalence, and canonicalization
    \item[\texttt{FundamentalDelaunay.h}] Fundamental Delaunay domains
\end{description}

\subsection{Dual Description Module (\texttt{src\_dualdesc})}

Specialized module for computing dual descriptions of polytopes on serial computers with optimizations for large-scale problems.

\subsubsection{Algorithms}
\begin{itemize}
    \item Serial dual description
    \item MPI-based parallel dual description
    \item Database-backed enumeration
    \item Balinski enumeration
\end{itemize}

\subsection{Copositivity Module (\texttt{src\_copos})}

This module implements tests for copositivity and complete positivity of matrices.

\subsubsection{Core Functionalities}
\begin{itemize}
    \item \textbf{Copositivity testing}: Determining if a matrix is copositive
    \item \textbf{Strict copositivity}: Testing for strict copositivity
    \item \textbf{Complete positivity}: Verifying complete positivity
    \item \textbf{Copositive kernel}: Computing the kernel in copositive cone
\end{itemize}

\subsection{Short Vector Module (\texttt{src\_short})}

Focuses on computing and enumerating short vectors in lattices.

\subsubsection{Main Programs}
\begin{description}
    \item[\texttt{SHORT\_GetShortestVector}] Finds shortest vectors in a lattice
    \item[\texttt{SHORT\_AutomorphismGroup}] Computes automorphism group from short vectors
    \item[\texttt{SHORT\_CheckPrimeRealizability}] Tests realizability over finite fields
    \item[\texttt{SHORT\_EnumerateCyclicCases}] Enumerates special cyclic cases
\end{description}

\subsection{Perfect Forms Module (\texttt{src\_perfect})}

Implements algorithms for perfect quadratic forms, which are critical points of the Hermite function.

\subsubsection{Functionalities}
\begin{itemize}
    \item Perfect form enumeration
    \item Perfect cone computations
    \item T-space analysis
    \item MPI-based parallel enumeration
\end{itemize}

\subsection{Indefinite Forms Module (\texttt{src\_indefinite})}

Handles computations with indefinite quadratic forms.

\subsubsection{Core Operations}
\begin{itemize}
    \item Automorphism group computation
    \item Isotropic plane stabilizers
    \item Canonical form computation
    \item Equivalence testing
\end{itemize}

\subsection{Lorentzian Module (\texttt{src\_lorentzian})}

Implements algorithms specific to Lorentzian (hyperbolic) quadratic forms.

\subsubsection{Algorithms}
\begin{itemize}
    \item \textbf{Vinberg algorithm}: Computing fundamental domains for reflection groups
    \item \textbf{Edgewalk algorithm}: Alternative approach using edge enumeration
    \item \textbf{Allcock algorithm}: Specialized 2-dimensional anisotropic cases
    \item \textbf{Coxeter dynamics}: Analysis of Coxeter diagrams and symbols
\end{itemize}

\subsection{Isotropy Module (\texttt{src\_isotropy})}

Focuses on finding isotropic vectors and reduction of quadratic forms.

\subsubsection{Functionalities}
\begin{itemize}
    \item Finding isotropic vectors
    \item Indefinite LLL reduction
    \item Diophantine approximation
    \item Determinant minimization
    \item Positivity testing
\end{itemize}

\section{Arithmetic Types}

The library supports multiple arithmetic types through templates, allowing users to choose the appropriate precision and performance trade-offs.

\subsection{Functional Types}

For exact computations over algebraic number fields:

\begin{description}
    \item[\texttt{Qsqrt2}] The field $\mathbb{Q}(\sqrt{2})$
    \item[\texttt{Qsqrt3}] The field $\mathbb{Q}(\sqrt{3})$
    \item[\texttt{Qsqrt5}] The field $\mathbb{Q}(\sqrt{5})$
    \item[\texttt{RealAlgebraic}] General real algebraic numbers (via continued fractions)
\end{description}

\textbf{Syntax}: Coefficients are written as: \texttt{1}, \texttt{1+x}, \texttt{x/4}, \texttt{-3-3x/4}, \texttt{3+(3*4)*x\^{}3}

\subsection{Speed Types}

For high-performance computations:

\begin{description}
    \item[\texttt{mpq\_class}] GMP rational numbers (recommended default)
    \item[\texttt{mpz\_class}] GMP arbitrary precision integers
    \item[\texttt{cpp\_rational}] Boost multiprecision rational (header-only)
    \item[\texttt{cpp\_int}] Boost multiprecision integer (header-only)
    \item[\texttt{SafeInt64}] Protected 64-bit integers with overflow detection
    \item[\texttt{Rational<SafeInt64>}] Safe rational arithmetic
\end{description}

\section{Compilation and Usage}

\subsection{Dependencies}

The following libraries are required:

\begin{itemize}
    \item \textbf{Eigen}: Linear algebra library (\url{http://eigen.tuxfamily.org/})
    \item \textbf{Boost}: C++ utility libraries (\url{http://www.boost.org/})
    \item \textbf{GMP}: GNU Multiple Precision Library (\url{https://gmplib.org/})
    \item \textbf{nauty}: Graph automorphism library (\url{https://pallini.di.uniroma1.it/})
\end{itemize}

\subsection{Building the Library}

The repository uses submodules. Clone with:

\begin{lstlisting}[language=bash]
git clone https://github.com/MathieuDutSik/polyhedral_common --recursive
\end{lstlisting}

Or if already cloned:

\begin{lstlisting}[language=bash]
./init.sh
\end{lstlisting}

Update submodules:

\begin{lstlisting}[language=bash]
./update.sh
\end{lstlisting}

Compilation details are provided in the Dockerfile at \texttt{docker\_files/Dockerfile}.

\subsection{Debug and Runtime Options}

Several environment variables control debugging and profiling:

\begin{description}
    \item[\texttt{DEBUG}] Enable debugging output and fast checks
    \item[\texttt{TIMINGS}] Print runtime information
    \item[\texttt{SANITY\_CHECK}] Perform consistency checks
    \item[\texttt{SANITY\_CHECK\_EXTENSIVE}] Perform expensive consistency checks
    \item[\texttt{KEY\_VALUE}] Print key-value pairs for optimization
    \item[\texttt{TRACK\_INFO}] Print tracking information
    \item[\texttt{METHOD\_COMPARISON}] Compare different method implementations
\end{description}

For granular control, use module-specific flags like \texttt{DEBUG\_LINEAR\_PROGRAM}.

\section{Testing}

\subsection{Test Organization}

\begin{description}
    \item[\texttt{CI\_tests/}] Continuous integration tests (5 min - 1 hour runtime)
    \item[\texttt{Exmpl\_Bench/}] Development and experimental tests
\end{description}

Tests are run via GitHub Actions with scheduled monthly execution to manage compute credits.

\section{Integration with Computer Algebra Systems}

The library can be interfaced with:

\begin{itemize}
    \item \textbf{GAP} (Group Algebra Programming)
    \item \textbf{Oscar} (Julia-based)
    \item \textbf{Sage} (Python-based)
\end{itemize}

Primary usage remains through CLI programs.

\section{Common Data Structures}

\subsection{Matrix Types}

\begin{lstlisting}
template <typename T>
class MyMatrix;  // Dense matrix

template <typename T>
class MyVector;  // Vector type
\end{lstlisting}

\subsection{Face Representation}

\begin{lstlisting}
using Face = boost::dynamic_bitset<>;  // Bitset for faces
\end{lstlisting}

Faces represent subsets of vertices or facets as bitsets for efficiency.

\subsection{Group Types}

Matrix groups are represented as vectors of generators:

\begin{lstlisting}
template <typename T>
using MatrixGroup = std::vector<MyMatrix<T>>;
\end{lstlisting}

\section{Performance Considerations}

\subsection{Group Exploitation}

A key optimization strategy is using group theory:
\begin{itemize}
    \item Orbit-based enumeration reduces redundant computation
    \item Canonical forms enable efficient isomorphism testing
    \item Stabilizer chains optimize group operations
\end{itemize}

\subsection{Memory Management}

The library uses:
\begin{itemize}
    \item Bitsets for compact face representation
    \item Move semantics for efficient memory transfer
    \item Template specialization for type-specific optimizations
\end{itemize}

\subsection{Parallelization}

Selected modules support MPI-based parallelization:
\begin{itemize}
    \item \texttt{LATT\_MPI\_ComputeDelaunay}
    \item \texttt{POLY\_MPI\_DualDesc}
    \item \texttt{PERF\_MPI\_EnumeratePerfectCones}
\end{itemize}

\section{Error Handling}

\subsection{Exception Types}

The library uses \texttt{TerminalException} for unrecoverable errors:

\begin{lstlisting}
throw TerminalException{1};
\end{lstlisting}

\subsection{Output Streams}

Functions accept \texttt{std::ostream\& os} parameters for flexible output:
\begin{itemize}
    \item Serial programs: pass \texttt{std::cerr}
    \item Parallel programs: pass process-specific streams
\end{itemize}

This prevents output mixing in parallel execution.

\section{Contributing and Development}

\subsection{Code Style}

\begin{itemize}
    \item Template-based generic programming
    \item Header-only when possible
    \item Clear separation of interface and implementation
    \item Extensive use of \texttt{const} correctness
    \item No trailing whitespace
\end{itemize}

\subsection{Adding New Functionality}

To add new functionality:

\begin{enumerate}
    \item Choose appropriate module or create new subdirectory
    \item Define template interfaces for arithmetic flexibility
    \item Add CI tests in \texttt{CI\_tests/}
    \item Document in header files
    \item Consider MPI parallelization for expensive operations
\end{enumerate}

\section{References and Further Reading}

\subsection{Theoretical Background}

\begin{itemize}
    \item \textbf{Lattice theory}: Gruber, M., \textit{Convex and Discrete Geometry}
    \item \textbf{Polytope theory}: Ziegler, G.M., \textit{Lectures on Polytopes}
    \item \textbf{Quadratic forms}: Cassels, J.W.S., \textit{Rational Quadratic Forms}
    \item \textbf{Vinberg algorithm}: Vinberg, E.B., \textit{Hyperbolic reflection groups}
\end{itemize}

\subsection{Repository}

\begin{itemize}
    \item GitHub: \url{https://github.com/MathieuDutSik/polyhedral_common}
    \item Issues and bug reports: GitHub issue tracker
\end{itemize}

\section{Appendix: Function Index}

\subsection{Group Module Functions}

\begin{description}
    \item[\texttt{write\_matrix\_group}] Write matrix group to file
    \item[\texttt{GetRationalInvariant}] Compute rational invariant
    \item[\texttt{LLLMatrixGroupReduction}] Apply LLL reduction to matrix group
    \item[\texttt{L1normMatrixGroup}] Compute L1 norm of matrix group
\end{description}

\subsection{Polyhedral Module Functions}

\begin{description}
    \item[\texttt{SelectRow}] Extract rows based on face
    \item[\texttt{SelectRowDropColumnFace}] Select rows and drop column
    \item[\texttt{TestFacetInequality}] Verify facet inequality
\end{description}

\subsection{Lattice Module Functions}

\begin{description}
    \item[\texttt{SelectShortestVector}] Find shortest vectors
    \item[\texttt{apply\_transformation}] Transform shortest vectors
    \item[\texttt{shortest\_get\_half}] Get half of shortest vectors
    \item[\texttt{EvaluationQuadForm}] Evaluate quadratic form
\end{description}

\section{Conclusion}

The Polyhedral Common Library provides a comprehensive, high-performance toolkit for computational mathematics research. Its modular design, support for multiple arithmetic types, and emphasis on group-theoretic optimization make it suitable for tackling challenging problems in polytope theory, lattice theory, and quadratic forms.

For questions, contributions, or bug reports, please visit the GitHub repository or contact the maintainer.

\end{document}
